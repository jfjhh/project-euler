\documentclass[final,letterpaper]{article}
\usepackage[english]{babel}
\usepackage[utf8]{inputenc}
\usepackage{amsmath}
\usepackage{amsthm}
\usepackage{dsfont}
\usepackage{nicefrac}
\usepackage{mathtools}


\title{Closed form of the difference between the square of the sum and the sum
of the squares of natural numbers}
\author{Alex Striff}
\date\today

\begin{document}
\maketitle

\section{Abstract}

The goal is to find a closed form of the following expression, the difference
between the square of the sum of natural numbers to $n$ and the sum of the
squares of the natural numbers to $n$.

\[
	{\left(\sum_{i=1}^{n}{i}\right)}^2 - \sum_{i=1}^{n}{i^2}
\]

\section{Triangle Numbers}

The easiest subproblem here is to find the sum of the first $n$ natural numbers:

\[
	T_n = \sum_{i=1}^{n}{i}
\]

The sum of $n$ numbers can be organized into $\nicefrac{n}{2}$ pairs of numbers,
pairing the first with the last, the second with the penultimate, and so on. The
sum of a pair will always be $n + 1$, so adding all $\nicefrac{n}{2}$ pairs of
$n + 1$ gives $\nicefrac{n (n + 1)}{2}$.

\begin{align*}
	T_n &= 1 + 2 + 3 + \cdots + (n - 2) + (n - 1) + n \\
	&= (1 + n) + (2 + (n - 1)) + (3 + (n - 2)) + \cdots
	+ (\frac{n}{2} + (n - \frac{n}{2})) \\
	&= \frac{n (n + 1)}{2}
\end{align*}

A geometric approach would be to plot $n$ rectangles in order with heights
ranging from $1$ to $n$ and constant widths of $1$. The area of these rectangles
is the $n$th triangle number. Drawing a line from $(0, 0)$ to $(n, n)$ would
make a right triangle with side length $n$ on the bottom and $n$ smaller right
triangles of side length $1$ on the top. The area of the larger triangle is
$\nicefrac{n^2}{2}$, and of any smaller triangle, $\nicefrac{1^2}{2}$. Because
there are $n$ smaller triangles, the area of all of the smaller triangles is
$\nicefrac{n}{2}$. Adding together the two areas, the above $\nicefrac{n (n +
1)}{2}$ is obtained.

\begin{align*}
	T_n &= \frac{1}{2}(n)(n) + n \cdot \frac{1}{2}(1)(1) \\
	&= \frac{n^2}{2} + \frac{n}{2} \\
	&= \frac{n (n + 1)}{2}
\end{align*}

To find the $n$th square of the sum term, square $T_n$. The expression can now
be simplified to:

\[
	{\left(\frac{n (n + 1)}{2}\right)}^2 - \sum_{i=1}^{n}{i^2}
\]

\section{Square Numbers}

The sum of the squares term is a bit more difficult. Upon writing down the first
few squares, a pattern emerges.

\begin{alignat*}{2}
	1^2  &= 1   &&= 1 \\
	2^2  &= 4   &&= 1 + 3 \\
	3^2  &= 9   &&= 1 + 3 + 5 \\
	4^2  &= 16  &&= 1 + 3 + 5 + 7 \\
	5^2  &= 25  &&= 1 + 3 + 5 + 7 + 9 \\
	6^2  &= 36  &&= 1 + 3 + 5 + 7 + 9 + 11 \\
	7^2  &= 49  &&= 1 + 3 + 5 + 7 + 9 + 11 + 13 \\
	8^2  &= 64  &&= 1 + 3 + 5 + 7 + 9 + 11 + 13 + 15 \\
	9^2  &= 81  &&= 1 + 3 + 5 + 7 + 9 + 11 + 13 + 15 + 17 \\
	10^2 &= 100 &&= 1 + 3 + 5 + 7 + 9 + 11 + 13 + 15 + 17 + 19 \\
	n^2  &= n^2 &&= \sum_{i=1}^{n}{2i - 1}
\end{alignat*}

So the $n$th square appears to be the sum of the first $n$ odd numbers. To use
this relationship later, it must be proved that this is true for all $n
\in \mathds{N}$.

\begin{proof}
	\begin{align*}
		\vdash 1^2
		&= 1 \\
		\vdash n^2
		&= \sum_{i=1}^{n}{2i - 1} \\
		&= \left(\sum_{i=1}^{n - 1}{2i - 1}\right) + (2i - 1) \\
		&= {(n - 1)}^2 + (2n - 1) \\
		{(n + 1)}^2
		&= {((n + 1) - 1)}^2 + (2 (n + 1) - 1) \\
		&= n^2 + 2n + 1 \\
		&= {(n + 1)}^2
	\end{align*}
	Therefore $n^2 = \sum_{i=1}^{n}{2i - 1}$ is true for all $\mathds{N}$ by
	induction.
\end{proof}

\section{Odd Numbers}

The task is now to sum squares by summing odd numbers. Expanding the sum:

\begin{align*}
	\sum_{i=1}^{n}{i^2}
	&= \sum_{i=1}^{n}{\sum_{k=1}^{i}{2k - 1}} \\
	&=\,1 \\
	&+\;1 + 3 \\
	&+\;1 + 3 + 5 \\
	&+\;1 + 3 + 5 + \cdots + (2n - 5) \\
	&+\;1 + 3 + 5 + \cdots + (2n - 5) + (2n - 3) \\
	&+\;1 + 3 + 5 + \cdots + (2n - 5) + (2n - 3) + (2n - 1) \\
	&= (n)(1) + (n - 1)(3) + (n - 2)(5) \\
	&+ \cdots + (3)(2n - 5) + (2)(2n - 3) + (1)(2n - 1) \\
	&= \sum_{i=1}^{n}{(n - i + 1)(2i - 1)}
\end{align*}

Any odd number $O_n = 2n - 1$ can be written as $O_n = 2n - 1 + (1 - 1)
= 2(n - 1) + 1$, i.e.\ the sum of $1$ and $(n - 1)$ $2$'s.

\begin{align*}
	\sum_{i=1}^{n}{i^2}
	&= (n)(1) \\
	&+ (n - 1)(1 + 2) \\
	&+ (n - 2)(1 + 2 + 2) \\
	&+ (n - 3)(1 + 2 + 2 + 2) \\
	&+ \cdots \\
	&= n \\
	&+ (n - 1) + (n - 1)(2) \\
	&+ (n - 2) + (n - 2)(2 + 2) \\
	&+ (n - 3) + (n - 3)(2 + 2 + 2) \\
	&+ \cdots \\
	&= T_n \\
	&+ 2(n - 1)(1) \\
	&+ 2(n - 2)(1 + 1) \\
	&+ 2(n - 3)(1 + 1 + 1) \\
	&+ \cdots \\
	&= T_n + 2\sum_{i=1}^{n}{(n-i)i} \\
	&= T_n + 2\sum_{i=1}^{n}{(ni - i^2)} \\
	&= T_n + 2n\left(\sum_{i=1}^{n}{i}\right) -
	2\left(\sum_{i=1}^{n}{i^2}\right) \\
	3\sum_{i=1}^{n}{i^2} &= T_n(2n + 1)\\
	\sum_{i=1}^{n}{i^2} &= T_n\left(\frac{2n + 1}{3}\right)
\end{align*}

So a closed form for the sum of the squares is now known.

\clearpage
\section{Conclusion}

Now the closed form for the square of the sum minus the sum of the squares of
the first $n$ natural numbers may be written.

\begin{align*}
	{\left(\sum_{i=1}^{n}{i}\right)}^2 - \sum_{i=1}^{n}{i^2}
	&= {T_n}^2 - T_n\left(\frac{2n + 1}{3}\right) \\
	&= T_n\left(\frac{n(n+1)}{2} - \frac{2n + 1}{3}\right) \\
	&= \frac{T_n}{6}(3n^2 - n - 2) \\
	&= \frac{n^2 + 2n + 1}{12}(3n^2 - n - 2) \\
	\\
	\Aboxed{%
		{\left(\sum_{i=1}^{n}{i}\right)}^2 - \sum_{i=1}^{n}{i^2}
		&= \frac{n^4}{4} + \frac{n^3}{6} - \frac{n^2}{4} - \frac{n}{6}
	}
\end{align*}

\end{document}

